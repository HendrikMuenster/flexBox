
\documentclass[final,leqno,onefignum,onetabnum]{article}

\usepackage{amsmath}
%\usepackage{amsthm}
\usepackage{amsfonts}

  \usepackage[caption=false]{subfig}
  \usepackage{graphicx}
  \usepackage{todonotes}
  \usepackage{tikz,tikz-cd,pgf}
\usepackage{listings}
\usepackage[numbered]{mcode} %for matlab code
  
  \usepackage{algorithm}
  \usepackage{algpseudocode}
   \usepackage{longtable}
  
  \usepackage{pgfplots}
  \pgfplotsset{compat=1.12}
  
  \usepackage{multirow}
  \newcommand{\R}{\mathbb{R}}
  \newcommand{\N}[1]{\mathbb{N}^{#1}}
  \newcommand{\1}[1]{\mathds{1}_{#1}}
  \newcommand{\st}{\qquad\text{s. t.}\qquad}

  \renewcommand{\liminf}[1]{\underset{#1}{\lim\inf}\;}
  \renewcommand{\limsup}[1]{\underset{#1}{\lim\sup}\;}
  
  \newcommand{\CX}{\mathcal{X}}
  \newcommand{\CY}{\mathcal{Y}}
  \newcommand{\CZ}{\mathcal{Z}}
  \newcommand{\dif}{\mathrm{d}}
  
  \DeclareMathOperator*{\argmin}{\arg \min}%
  \DeclareMathOperator*{\argmax}{\arg \max}%
  \newcommand*\cpp{C\kern-0.2ex\raisebox{0.4ex}{\scalebox{0.8}{+\kern-0.4ex+}}}
  \newcommand{\tabitem}{~~\llap{\textbullet}~~}

  %For pseudocode
 \newcommand*\Let[2]{\State #1 $\gets$ #2}
  
\title{A \textbf{\underline{Flex}}ible Primal-Dual Tool\textbf{\underline{Box}} \\\large Manual}
\author{Hendrik Dirks}


\begin{document}
	
\maketitle


\section{How-To}

\section{Arbitrary Operators}
\textbf{Classes}: \textit{L1operatorIso, L1operatorAniso, L2operator, frobeniusOperator}\\
Adding an arbitrary operator in some norm is simple. Let us assume we some term $\alpha \|Ku\|_{1,2}$ with 
\begin{align*}
	K=\begin{pmatrix}K_1&K_2\\K_3&K_4\end{pmatrix}, u=\begin{pmatrix}u_1\\u_2\end{pmatrix},
\end{align*}
where the norm $\|\cdot\|_{1,2}$ refers to the isotropic L1-norm. We can insert this term into \textbf{FlexBox} by calling
\begin{lstlisting} 
%Begin: Code example
main.addTerm(L1operatorIso(alpha,2,{K_1,K_2,K_3,K_4}),[numberU_1,numberU_2]);
%End: Code example
\end{lstlisting}
The operator $K$ has to be specified row-wise in a cell-array. The second argument $2$ tells the toolbox that every two elements in the cell-array correspond to one row. Please note that empty blocks in $K$ have to be specified as empty sparse matrices.


	
\listoftodos

%\bibliographystyle{plain}
%\bibliography{references}
	
\end{document}